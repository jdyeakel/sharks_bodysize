\documentclass[ucm,12pt]{ucletter}

\setlength\parindent{1cm}

\usepackage{setspace}
\usepackage{graphicx}

%\name{Justin D. Yeakel}
% \telephone{(209)~285-9571}
\email{skim380@ucmerced.edu\\ jdyeakel@gmail.com}

\begin{document}
\begin{letter}{
  \\The Royal Society\\
   6-9 Carlton House Terrace\\
  London SW1Y 5AG\\
  United Kingdom\\ \\
    \centerline{\bf{Re: Decoding the dynamics of dental distributions: insights from shark demography and dispersal}}
    \vspace{-5mm}
}


\opening{To the Editorial Board at the \emph{Proceedings of the Royal Society B - Biological Sciences},}

\setstretch{1.25}

%Intro
Please find attached the manuscript entitled \emph{Decoding the dynamics of dental distributions: insights from shark demography and dispersal} co-authored by Sora L. Kim, Justin D. Yeakel, Meghan A. Balk, Jaelyn J. Eberle, Sarah S. Zeichner, Dina Fieman, and J\"urgen Kriwet, which we would like to submit for publication at the \emph{Proceedings of the Royal Society B}. 

Shark teeth are one of the most abundant vertebrate fossils in the paleontological record, indirectly documenting the ecology and evolution of both extinct and contemporary shark species spanning the last 400 million years of their presence in marine systems.
Yet, despite their prolific abundance, these accumulations of dentition are enigmatic, typically used to identify species occurrence and morphological change over evolutionary timescales.
% Within species, the distributions of shark dentition reveal unique geometries: some are skewed to small or large tooth sizes, some are bimodal, and some appear normally distributed.
Here we argue that in some cases shark dental distributions, and by extension body size distributions, may communicate important information concerning their life history, in particular their dispersal strategies.
We examine the shapes of both contemporary and Eocene sand tiger dental distributions, and investigate their differences using a mechanistic simulation of shark population dynamics to reconstruct expected distributions.
% Our simulation assumes that shark individuals disperse between a nursery (estuarine) and adult (pelagic) habitats, where teeth accumulate in both sites as they grow and disperse.
% By comparing empirical dental distributions against those simulated with our model, we then attempt to elucidate characteristics of shark life history governing both Eocene and modern sand tiger life histories.


The results of our investigation reveal that 
\begin{itemize}
    \item Our population simulation is capable of reproducing the large range of observed dental distribution shapes as a function of varying dispersal strategies between a nursery and adult site
    \item Our framework accurately captures known life history characteristics of contemporary sand tiger populations, and is able to correctly characterize the likely roles of Eocene habitats based on paleontological reconstructions
    \item By linking dental distribution shape to specific shark dispersal strategies with our simulation framework, our results point to the importance of the presence and role of shark nurseries among sand tiger populations from the Eocene to the present
\end{itemize}

Support for the central role of sand tiger nurseries across tens of millions of years is important for both understanding the past lives of extinct species, but also for conserving future populations, as the importance of shark nurseries for conservation efforts is controversial.
Our work integrates a number of perspectives and sub-disciplines in paleontology, ecology, and shark science that should appeal to a broad readership. We expect this submission to be of particular interest to those using accumulations of teeth to reconstruct the lives and lifestyles of fossil species, as well as those studying the life histories of contemporary shark species.
Overall we believe that our approach offers a unique perspective into deriving ecological insights from such abundant fossil remains, and identifies aspects of shark life history that may be vital for successful conservation of shark species in a changing world.


\vspace{0mm}

\singlespacing
\closing{Sincerely,\\
\fromsig{
\includegraphics[scale=0.14]{signature_sk.jpg}
\includegraphics[scale=0.2]{signature_jy.jpg}
}\\
\fromname{
Sora Kim,\\
Justin D. Yeakel,\\
University of California, Merced}
}



\end{letter}
\end{document}
