%% Author_tex.tex
%% V1.0
%% 2012/13/12
%% developed by Techset
%%
%% This file describes the coding for rsproca.cls

\documentclass[]{rsos}%%%%where rsos is the template name

\usepackage{wasysym}


%%%% *** Do not adjust lengths that control margins, column widths, etc. ***

%%%%%%%%%%% Defining Enunciations  %%%%%%%%%%%
\newtheorem{theorem}{\bf Theorem}[section]
\newtheorem{condition}{\bf Condition}[section]
\newtheorem{corollary}{\bf Corollary}[section]
%%%%%%%%%%%%%%%%%%%%%%%%%%%%%%%%%%%%%%%%%%%%%%%


\begin{document}

%%%% Article title to be placed here
\title{Can we sneak a Jaws reference here? like "It's a squal us" or "Smile" or something about a need for a "bigger boat" or something about "going fishing" or a "toothy problem". OR: Show me your teeth: determining demography.... what about: Decoding the dynamics of dental distributions: insights from shark demography and dispersal}

% Decoding the dynamics of dental distributions with demographic 

\author{%%%% Author details
Sora Kim$^{1}$, Justin D. Yeakel$^{1}$, Jaelyn Eberle$^{2}$, Sarah Zeichner$^{3}$}
% and X. Third author$^{3}$}%

%%%%%%%%% Insert author address here
\address{$^{1}$School of Natural Science, University of California Merced,
$^{2}$Department of Geological Sciences and Museum of Natural History, University of Colorado,
$^{3}$Division of Geological and Planetary Sciences, California Institute of Technology}


%%%% Subject entries to be placed here %%%%
\subject{paleontology, taphonomy, ecology, body size}

%%%% Keyword entries to be placed here %%%%
\keywords{sharks!, sharks!, sharks!}

%%%% Insert corresponding author and its email address}
\corres{Sora Kim\\
\email{skim380@ucmerced.edu}}

%%%% Abstract text to be placed here %%%%%%%%%%%%
\begin{abstract}
Brilliant words.
\end{abstract}
%%%%%%%%%%%%%%%%%%%%%%%%%%%


\maketitle

\section{Introduction}
% Extant sharks occupy a wide range of habitats varying from fresh, brackish, and saline waters as well as all oceanic basins except the Southern Ocean of Antarctica.

Sharks have been a cornerstone of oceanic communities for hundreds of millions of years, a rare constant in a sea of change.
The enormous spatial and temporal dominance of shark species suggests considerable ecological plasticity, which has likely contributed to their evolutionary success and may be the key to understanding the ongoing and future effects of climate change.
Documenting the success of sharks as marine predators has been a trail of fossilized teeth, accumulating in ocean sediments and indirectly recording their ecological variability as well as the oceanic conditions in which they lived.
While shark teeth are the most abundant vertebrate fossil, with a record spanning 400 million years, the dynamics giving rise to these dental distributions -- in particular the interacting effects of shark ecology and the environment -- is not well characterized or quantified.
% These ecological and evolutionary traits hint that sharks are resilient to climate change; however, an integrated risk assessment for sharks and rays in the Great Barrier Reef suggests those in coastal areas with freshwater inputs are most vulnerable due to changes in temperature, salinity, and ocean circulation \cite{Chin2010}.
% Inferences to ecological and environmental processes are difficult to deduce from fossils because the large spatial and temporal distribution of the recorded data. 
% Fossil shark teeth are the most abundant vertebrate fossil and often the only element for paleoecological studies since their skeletal tissue rarely withstand fossilization.

The geologic record is prolific with fossil shark teeth and offers potential past ‘experiments’ when sharks survived past periods of climate change. 
These specimens contain insights to ecology and environment within their morphological and geochemical composition. 
For example, oxygen isotope analysis of shark enameloid indicates tolerance across a large temperature and salinity range \cite{Kim2014d, Kim2020}
while collections of teeth show how community composition or morphology changed during environmental change (Sibert et al. 2018; Sibert et al. 2020).
Here, we seek to delve further into ecology by exploring how body size distributions project ecological interactions and traits as well as environmental condition in sharks.
% INSERT: something related to examples of overcoming or altering distribution to salinity / oxygen change.


Body size is a biological indicator often used to determine ontogeny and energy balance in mammals and fish. 
% in sharks, body size is related to tooth crown height where, when comparing the same tooth position, smaller teeth are associated with smaller individuals \cite{Shimada2002, Shimada2004, Shimada2007, Shimada2020, Villafana2020}.
Body size distributions of species within a geographic area are controlled by a combination of biotic interactions [e.g., in plants (Muller-Landau et al. 2006; West et al. 2006)] and resource acquisition (Kerr and Dickie 2001; Ernest et al. 2003).
Less known are controls on body size distributions of a species across its geographic range.
In fish, in addition to resources and biotic controls, temperature plays an important role in determining growth speed and asymptotic size. 
One way to probe the interaction between biology and environment when interpreting  body size distributions is to look at theoretical models that factor life history and temperature.
These dynamic model outputs can be compared to empirical data to infer processes and mechanisms that shape body size distributions.
One fascinating period to explore shark paleobiology is the greenhouse climate during the Eocene (56-34My) when sharks were distributed across a large latitudinal range spanning the Arctic \cite{Eberle2012, fieman2016comparing} to Southern Ocean \cite{Long1992, Kriwet2016}.
This geologic time is often invoked as a deep-time analogue for future climate change and may provide insight to the ecological response of sharks in variable environmental conditions.  

During the Eocene, the Arctic Ocean and Southern Ocean of Antarctica were very different from today with ice-free waters and temperate flora and fauna, but both ecosystems supported shark ecosystems \cite{Kriwet2005, Reguero2012, Padilla2014, Kriwet2016, Engelbrecht2019}(Figure 1).
These two high latitude localities had differences: the Arctic Ocean was brackish with reduced salinity \cite{Waddell2010,Kim2014} and low shark diversity \cite{Greenwood2010,Padilla2014} whereas the Southern Ocean was a fully marine, near shore habitat \cite{Ivany2008} with a diverse fossil shark record \cite{Long1992,Kriwet2016}.
Despite environmental and assemblage differences of these two high latitude localities, they share Sand Tigers \emph{Striatolamia macrota} Agassiz, 1843 and \emph{Carcharias macrota} \cite{Kriwet2005, Reguero2012, Padilla2014, Kriwet2016, Engelbrecht2019}, which share many characteristics and are synonymized by some \cite{purdy1998chondrichthyan}.
Further, ancient Sand Tigers were also abundant at lower latitudes and are preserved in Eocene sediments throughout the Gulf of Mexico \cite{Westgate}.
We chose two mid-latitude localities because their community assemblage resembled the those in high latitudes.
The Tuscahoma Fm. suggests a reduced salinity \cite{ingram1991tuscahoma, Beard2009} similar to the Eocene Arctic at Banks Island whereas Stone City Fm. has a diverse assembly and reflects a pelagic, marine community \cite{Breard1999, Westgate, harding2014mineralogy} (and personal correspondence with C.J. Flis, Oct. 2016) on par with the Eocene Antarctic La Meseta Fm. at Seymour Island. 
% In contrast, the Stone City Fm. at Whiskey Creek Bridge is dominated by marine fauna and sediments indicate a sheltered, 20-60m depth micro-tidal sea subject to repetitive hurricane-intensity storm disturbance . 
 The large latitudinal distribution of Sand Tigers is almost continuous in the Eocene and fossil teeth provide biologically relevant data that allow comparisons among populations and insight to ecological plasticity.

%Body size is an often measured biological trait because it reflects an organism's energy balance, for which there are different demands related to ontogeny and environment. 
% Body size is thought to increase with latitude for endothermic taxa (i.e., birds and mammals); however, there is mixed evidence for ectotherms and even differing patterns at the generic and species level. 
% A review of freshwater fishes suggests an inverse Bergmann’s Rule \cite{Belk2002}, but this hypothesis is difficult to test for sharks today given the low diversity in the Arctic Ocean (i.e., only Greenland sharks) and lack of species in today’s Southern Ocean. 
% [MAB NOTE: I think we stay away from Bergmann's Rule only because this paper doesn't really test it and because there isn't a temperature gradient.] SLK: deleted that bit
In many modern sharks, the nursery sites and feeding areas are distinct. 
For example, Great White Sharks travel thousands of miles between the "White Shark Cafe" and coastal nursery sites \cite{Jorgensen2010}. Recently, a great hammerhead nursery was identified proximal to downtown Miami, FL \cite{Macdonald2021}.
The distribution of body sizes at the nursery sites and feeding sites are likely different.
Declining nursery area has been invoked as extinction mechanisms for sharks (e.g., fossil shark Pimiento et al. 2010, modern shark Castro 1987), although there is little support for the protection of nurseries reducing population declines in modern sharks (Kinney and Simpfendorfer 2009).
Presumably, ancient sharks had similar behavior and utilized nursery areas.
Do the body size distributions indicate ecological traits, such as migration rates? 
We tracked growth and movement between a nursery and adult site with a  population dynamic simulation to explore the interactive effects of temperature, migration, and body size. 
Then, compared simulation outputs to empirical body size data from modern Sand Tigers and Eocene Sand Tigers from four fossil localities. 

Sharks are heterodont and tooth morphology is indicative of position, similar to mammals. 
Studies with extant shark species find robust relationships between a single position's tooth crown height and total body length \cite{Shimada2002, Shimada2004, Shimada2007}.  
In localities with abundant fossil shark teeth, robust body size distributions are used to determine habitat use (i.e., nursery grounds) and maximum size \cite{Pimiento2010, Shimada2020, Villafana2020}; to date, we are unaware of studies that couple these body size distributions with theoretical ecology that can probe the limits imposed by extrinsic environmental factors.
Here, we selected four Eocene fossil sites at high and mid-latitudes with differing degrees of coastal proximity based on their fossil assemblage and sedimentology (Figure 1).
Then, we explored the influence of temperature on growth and migration on  juvenile and adult interaction with a dynamic population model that tracked body size distribution of sharks at two sites.
Finally, we compared the empirical tooth size distributions from the Eocene to simulation outputs to discern possible migration parameters and juvenile vs. adult populations.


\section{Methods}

%\subsection{Geologic Setting}
%\underline{Banks Island, NWT Canada}
%The fossil shark teeth from the Arctic were discovered as float on the unconsolidated Cyclic Member sands of the Eureka Sound Formation close to the Muskox and Eames rivers within Aulavik National Park on northern Banks Island, Northwest Territories (NWT), Canada (~74o N). 
%Researchers should request the exact coordinates from the Canadian Museum of Nature, Ottawa, Canada. 
%The fossil shark teeth localities near the Muskox and Eames rivers are Eocene in age. This is based on pollen samples, mammalian biostratigraphy, and a zircon produced date of 〖52.6〗\pm 1.9 Ma \cite{Miall1979, Sweet2012, Reinhardt2010, Hopkins1974, Hopkins1975}. 

%\underline{Seymour Island, Antarctica}
%The Antarctic shark teeth are from the La Meseta Formation on Seymour Island, 100 km east of the Antarctic Peninsula at 64°17’S, 56°45’W \cite{Sadler1988, Ivany2008, Kriwet2005, Kriwet2016, Engelbrecht2017, Engelbrecht2017b, Engelbrecht2017c, Engelbrecht2017d, Engelbrecht2019}. 
%Previous studies estimate that La Meseta Fm. spans Early – early Middle Eocene to Late Eocene, with TELMs 2-5, which contain \emph{S. macrota} teeth, spanning ~early Middle Eocene to Late Middle Eocene \cite{Long1992, Stilwell1992, Reguero2012, Kriwet2016}    
%Previous studies have determined that the La Meseta Fm. depositional environment was estuarine \cite{Marenssi1994, Ivany2008, Amenabar2020}.

%\underline{Whiskey Bridge, Burleston County, TX USA}
%The shark teeth from Texas were found in the Stone City Formation (~41.8Ma), which is located on the south bank of the Brazos River in Burleston County, TX, 18.5km west of Bryan, Brazos County, TX \cite{Breard1999, Westgate} (and personal correspondence with C.J. Flis, Oct. 2016). 
%Stone City Fm.’s late Middle Eocene age was determined by biostratigraphy and radiometric dating of nearby fossil ash (Heintz et al., 2015; personal correspondence with C.J. Flis, Oct. 2016).
%Previous studies of Stone City Fm. sediment deposition suggest a sheltered, 20-60m depth micro-tidal sea subject to repetitive hurricane-intensity storm disturbance \cite{Breard1999, harding2014mineralogy} (and personal correspondence with C.J. Flis, Oct. 2016).  

%\underline{Red Hot Truck Stop, Meridian, Mississippi}
%The shark teeth found in eastern Mississippi were discovered in the unconsolidated sands of the T4 Channel Sand at the top of Tuscahoma Fm. at the Red Hot Truck Stop locality (Carnegie Museum or CM 517), near Interstate 20, in the NW corner, of the NW ¼, of the NE ¼, of Section 20, T6N, R16E, Lauderdale County, Mississippi \cite{ingram1991tuscahoma}. 
%At the base of the T4 Sand is a lag deposit that preserved the vertebrate teeth and fossil fragments (Ingram, 1991). The upper Tuscahoma T4 sand was dated to be 55-+ 1.4 Ma \cite{mancini1995geochronology} using potassium-argon (K-Ar) radiometric age determination.
%The lithology of the Tuscahoma Formation and T4 Sand is consistent with that of a large-scale, fluvial-dominated deltaic system \cite{Beard2009}. 

\subsection{Tooth Identification and Measurement}
Species of sharks in the fossil record are largely identified by their tooth morphology \cite{Cappetta2012} due to the poor preservation of cartilaginous skeletons. 
\emph{Striatolamia macrota} teeth are identified by their strong striations on the lingual side of the tooth and a smooth labial side \cite{Cappetta2012}. 
The anteriors (A1-2 and a1-2) are recognized by their long and narrow shape, compared to the laterals and posteriors that have a short, blade-like appearance (Cunningham, 2000). 
The anterior teeth have an acute angle between the two roots and have two small lateral cusplets \cite{Padilla2014, Cappetta2012}. 
This tooth position was chosen because their large size and distinct morphology compared to other tooth positions within the jaw. 
We measured anterior tooth height from the enameloid base to the blade tip with digital calipers to an accuracy of 0.1mm. 
The labial and lingual sides of the tooth, and the maximum width was also measured and recorded. 
The labial side of the tooth is adjacent to the cheek of the shark, and the lingual is that side adjacent to the tongue (Fig. 6) \cite{Cappetta2012}.  
Every seventh tooth was re-measured for 0.3mm accuracy. 

Sharks replace teeth throughout their lifetime with a conveyor belt-like system. 
In many modern sharks, including Sand Tigers, teeth within an individual position scale with body size \cite{Shimada2002, Shimada2004, Shimada2007, Shimada2020}. 
We used anterior tooth height as a proxy for body size and interpret population size distribution for \emph{Striatolamia macrota} in each locality. 
Sand tiger teeth from Banks Island are archived at the Canadian Museum of Nature (Ottawa, ON Canada), Seymour Island archived at the University of California Museum of Paleontology (UCMP; Berkeley, CA USA) and Paleontological Research Institute (PRI; Ithaca, NY USA), Red Hot Truck Stop locality archived at the Carnegie Museum of Natural History (CM; Pittsburgh, PA), and Whiskey Bridge locality archived at the Whiteside Museum of Natural History (WMNH; Seymour, TX). 
Teeth from Banks Island and Red Hot Truck Stop were measured by DMF in 2015 – 2016 and teeth from Seymour Island and Whiskey Bridge were measured by SSZ in 2015. 

To ground truth the population model, we also used total length measurements from an extant \emph{Carcharias taurus} population in Delaware Bay, USA. 
We transformed total length measurements from the 2012 tagging season provided by Dr. Dewayne Fox at  Delaware State University to anterior tooth crown height based on Eqn. XX. 
Given the extensive knowledge of this extant population's movement patterns, this body size distribution provided a qualitative metric for the model's accuracy.



%Here is some more stuff.
%Sometimes we use $\delta^{13}{\rm C}$ notation which is in $\permil$ \cite{Brown2004}.

% \subsection{Population model}

% To explore specific ecological mechanisms that may be responsible for the observed tooth size distributions, we employed a process-based model that allows us to incorporate likely physiological and ecological constraints likely influencing paleontological shark populations.
% We constructed a two-site size-class model that tracks female shark populations over time, where one of the two sites is designated a nursery.
% We considered four key dynamics influencing changes in population size for both sites: reproduction, somatic growth, mortality, and migration between sites.
% Sites were assumed to be a variable distance apart, where seasonal fluctuations in temperature reached site-specific minimum (winter) and maximum (summer) extremes, allowing us to consider the effects of locations farther from and closer to the equator.

% % \noindent \textbf{Reproduction and mortality:} 
% In our framework, reproduction takes place only at the nursery site, whereas mortality occurs at both sites.
% The per-capita reproductive rate $r$ was thus set to $r=0$ at the adult site, and $r = 0.47 \times 10^{-7}$ female pups per second (Cortes 1996) at the nursery site, independent of time of year or water temperature. 
% The per-capita mortality rate was assumed to be constant for both adult and juvenile sites, set at $\mu = 5.71 \times 10^{-9}$ individuals per second, corresponding to  (Schindler 2002).

% % \noindent \textbf{Somatic growth:}
% Shark individuals were assumed to increase in size following the growth trajectory described by West et al. \cite{West:2001bv} as a function of metabolic rate.
% Partitioning of metabolism $B$ between growth and maintenance can be used to derive a general equation for both the
% growth trajectories and rates of organisms ranging from bacteria to
% metazoans
% \cite{West:2001bv,moses2008rmo,gillooly2002esa,hou,Kempes:2012hy}. 
% This relationship is derived from the balance condition 
% $B_{0}m^{\eta}=E_{m}\dot{m}+B_{m}m\,,$
% % \begin{eqnarray}
% % \label{balance}
% % B_{0}m^{\eta}=E_{m}\frac{dm}{dt}+B_{m}m\,,
% % \end{eqnarray}
% \cite{West:2001bv,moses2008rmo,gillooly2002esa,hou,Kempes:2012hy} where $E_{m} = 5774$ (J g${}^{-1}$) is the energy needed to synthesize a unit of mass \cite{moses2008rmo,hou,Pirt1965,Heijnen1981}, $B_{m}$ is the metabolic rate to support an existing unit of mass, $B_0$ (W g${}^{-3/4}$) is the metabolic normalization constant, and $m$ (g) is the mass of the organism at any point in its development.
% % This balance has the general solution \cite{bettencourt,Kempes:2012hy}
% % \begin{eqnarray}
% % \label{m1}
% % \left(\frac{m\left(t\right)}{M}\right)^{1-\eta}\!=1\!-\!\left[1\!-\!\left(\frac{m_{0}}{M}\right)^{1\!-\!\eta}\right]e^{-a\left(1\!-\!\eta\right)t/M^{1-\eta}},
% % \end{eqnarray}
% % where, for $\eta<1$, $M=(B_{0}/B_{m})^{1/(1-\eta)}$ is the asymptotic mass, $a=B_{0}/E_{m}$, and $m_0$ is mass at birth.  
% % We now use this solution to define the timescale for reproduction and recovery from starvation (Fig.~\ref{fig:growth}; see \cite{moses2008rmo} for a detailed presentation of these timescales). 
% The time that it takes to reach a proportion $\epsilon$ of the asymptotic adult mass $M$ is given by the timescale
% \begin{equation}
% \label{t1}
% \tau\left(\epsilon\right) = \ln\left[\frac{1-\left(m_{0}/M\right)^{1-\eta}}{1-\epsilon^{1-\eta}}\right]\frac{M^{1-\eta}}{a\left(1-\eta\right)},
% \end{equation}
% given $a=B_0/E_m$, the scaling exponent $\eta = 3/4$, and $m_0$ is the individual's mass at birth \cite{Yeakel2018}. 
% Because the sharks that we are considering here are assumed to be ectotherms, we incorporate a temperature-dependence for metabolic parameters (Brown 2004), such that $B_0 = {\rm e}^C{\rm e}^{-E/kT}$, where the normalization constant $C=18.47$ for fish, the activation energy $E=0.63$ (eV), Boltzmann's constant $k=8.6173\times 10^{-5}$ (eV Kelvin ${}^{-1}$), and $T$ (Kelvin) is temperature \cite{Brown2004}.
% Accordingly, shark individuals grow more quickly in warm environments, reaching the asymptotic mass $M$ at a younger age.
 
% % For the time to reproduce, $t_{\lambda}=\tau\left(\epsilon_{\lambda}\right)$, where $\epsilon_{\lambda}$ is the fraction of the asymptotic mass where an organism is reproductively mature and should be close to one (typically $\epsilon_{\lambda}\approx0.95$; \cite{West:2001bv}). The growth rate is then given by $\lambda=\ln\left(\upsilon\right)/t_{\lambda}$ where $\upsilon$ is the number of offspring produced, and for any constant value of $\epsilon_{\lambda}$, this rate will scale as $\lambda\propto M^{\eta-1}$ for $M\gg m_{0}$ \cite{West:2001bv,moses2008rmo,gillooly2002esa,hou,Kempes:2012hy}.


% % \noindent \textbf{Migration:}
% In our model we consider two sites: a site where adult females give birth to juveniles (the nursery), and a site where individuals grow to their adult size (the adult site).
% %Distance and velocity on migration rate d
% The maximal migration rate is assumed to be a function of the distance between the nursery and the adult site, such that $d_{\rm max} = v/\delta$, where velocity $v=1$ meters/second and distance $\delta$ (meters) is varied. 
% If the nursery is very close to the adult site, individuals will migrate between sites at a higher rate, such that both sites will be well mixed.
% As the distance between sites increases, the extent to which the sites are mixed declines.
% If the distance is large enough, individuals may grow in size as they travel between sites.

% Migrations of adults to the nursery, and of both juveniles and adults from the nursery to the adult site, are considered separately because we assume these migrations operate according to two different drivers.
% When newborns of size $m_0$ are born in the juvenile site, we assume they begin migrating to the adult site once they reach a size $m_j = (1/4)M$.
% Assuming an asymptotic mass $M=110$ Kg, $m_0 = 8$ Kg, and $m_j = 27$ Kg, the time that it takes for this to occur is given by Eq. \ref{t1} and is $\tau = XX$.
% Importantly, this migration is unlikely to be exact, and migration might occur at sizes smaller or larger than $m_j$.
% To incorporate this additional flexibility into the juvenile migration, we assume that the migration rate $d$ is size-dependent, varying from $d=0$ below $m_0$ and increasingly sigmoidally to $d=d_j^{\rm max}$.
% The \emph{mass-dependent} juvenile migration to the adult site is thus described as
% \begin{equation}
%     d_j(m) = \frac{d_j^{\rm max}}{1 + {\rm e}^{-1/\xi_j}},
% \end{equation}
% where $\xi_j$ describes the flexibility of the size-dependent migration: higher values of $\xi_j$ means that there is more variability in migrating juvenile size above and below $m_j$.

% %From adult site to nursery (a function of time)
% We assume that females in the adult site migrate to the juvenile site seasonally.
% Given a day of the year where the migration rate is maximized $d_a^{\rm max}$ on day $t_{\rm peak} = 180$ out of 365 days in a year, the migration rate decays before and after this peak in a Gaussian manner.
% The \emph{time-dependent} adult migration from the adult site to the juvenile site is thus described as
% \begin{equation}
%     d_a(t) = d_a^{\rm max}{\rm e}^{\frac{-(t - t_{\rm peak})^2}{2\xi_a^2}}.
% \end{equation}
% Here, $\xi_a$ describes the flexibility of the time-dependent migration: higher values of $\xi_a$ mean that there is more variability with regard to when adults migrate to the nursery. 


% %Tooth drops
% Because we aim to understand the shapes of tooth distributions from the perspective of shark population dynamics, we must simulate the loss of teeth over time in both nursery and adult sites.
% We focus only on the loss of upper and lower A1 teeth to reflect those used to build the empirical distributions.
% To simulate accumulated tooth distributions, we assumed a tooth loss rate of one upper and lower tooth every 40 days, or $5.79\times 10^{-7}$ teeth/second.
% Sharks of different body sizes drop teeth of different sizes, so shark life-history is directly captured by accumulated teeth of differing sizes.
% The migrational dynamics serve to mix or segregate teeth dropped by reproducing adults and recently born offspring in the nursery site from the larger individuals in the adult site.


% %Parameterization

% %Justification
% Examining equatorial (warm, low-seasonality) vs. non-equatorial (cold, high-seasonality) environments allowed us to examine the effects of different temperatures regimes on accumulated tooth distributions, we also examined a system reflecting expected temperature regimes and seasonality of the Eocene Ocean.
% We consider the dynamics of three different migration systems: 
% \emph{i}) a population with both adult and nursery sites far from the equator (cold temperatures, high seasonality),
% \emph{ii}) a population with both adult and nursery sites near the equator (warm temperatures, low seasonality), and
% \emph{iii}) a population with the adult site near the equator (warm, low seasonality), and the juvenile site far from the equator (cold, high seasonality).
% [More about Eocene Ocean]
% In addition we considered the effects of three different distances separating juvenile from adult sites, at 200, 400, and 650 Km.
\subsection{Population model}

To explore specific ecological mechanisms that may be responsible for the observed tooth size distributions, we employed a process-based model that allows us to incorporate likely physiological and ecological constraints influencing paleontological shark populations.
We constructed a two-site size-class model that tracks female shark populations over time, where one of the two sites is designated a nursery.
We considered four key dynamics influencing changes in population size for both sites: reproduction, somatic growth, mortality, and migration between sites.
Sites were assumed to be a variable distance apart, where seasonal fluctuations in temperature reached site-specific minimum (winter) and maximum (summer) extremes, allowing us to consider the effects of locations farther from and closer to the equator.
By simulating shark population dynamics we tracked changes in population size structure. 
A comparison of simulated body size distributions against those observed from different environments thus allows us to propose specific ecological mechanisms giving rise to observed features in empirical size distributions from site to site.
Because there are not large body size differences between males and females (REF), our model is restricted to the migration and reproductive dynamics of females.

% \noindent \textbf{Reproduction and mortality:} 
In our framework, reproduction takes place only at the nursery site, whereas mortality occurs at both sites.
The per-capita reproductive rate $r$ was thus set to $r=0$ at the adult site, and $r = 0.47 \times 10^{-7}$ female pups per second (Cortes 1996) at the nursery site, independent of time of year or water temperature. 
The per-capita mortality rate was assumed to be constant for both adult and juvenile sites, set at $\mu = 5.71 \times 10^{-9}$ individuals per second (Schindler 2002).

% \noindent \textbf{Somatic growth:}
Shark individuals were assumed to increase in size following the growth trajectory described by West et al. \cite{West:2001bv} as a function of metabolic rate.
Partitioning of metabolism $B$ between growth and maintenance can be used to derive a general equation for both the
growth trajectories and rates of organisms ranging from bacteria to
metazoans
\cite{West:2001bv,moses2008rmo,gillooly2002esa,hou,Kempes:2012hy}. 
This relationship is derived from the balance condition 
$B_{0}(T)m^{\eta}=E_{m}\dot{m}+B_{m}(T)m\,,$
% \begin{eqnarray}
% \label{balance}
% B_{0}m^{\eta}=E_{m}\frac{dm}{dt}+B_{m}m\,,
% \end{eqnarray}
\cite{West:2001bv,moses2008rmo,gillooly2002esa,hou,Kempes:2012hy} where $E_{m} = 5774$ (J g${}^{-1}$) is the energy needed to synthesize a unit of mass \cite{moses2008rmo,hou,Pirt1965,Heijnen1981}, $B_{m}(T)$ is the temperature (T)-dependent metabolic rate to support an existing unit of mass, $B_0(T)$ (W g${}^{-3/4}$) is the temperature (T)-dependent metabolic normalization constant, and $m$ (g) is the mass of the organism at any point in its development.
% This balance has the general solution \cite{bettencourt,Kempes:2012hy}
% \begin{eqnarray}
% \label{m1}
% \left(\frac{m\left(t\right)}{M}\right)^{1-\eta}\!=1\!-\!\left[1\!-\!\left(\frac{m_{0}}{M}\right)^{1\!-\!\eta}\right]e^{-a\left(1\!-\!\eta\right)t/M^{1-\eta}},
% \end{eqnarray}
% where, for $\eta<1$, $M=(B_{0}/B_{m})^{1/(1-\eta)}$ is the asymptotic mass, $a=B_{0}/E_{m}$, and $m_0$ is mass at birth.  
% We now use this solution to define the timescale for reproduction and recovery from starvation (Fig.~\ref{fig:growth}; see \cite{moses2008rmo} for a detailed presentation of these timescales). 
The time that it takes to reach a proportion $\epsilon$ of the asymptotic adult mass $M$ is given by the timescale
\begin{equation}
\label{t1}
\tau\left(\epsilon\right) = \ln\left[\frac{1-\left(m_{0}/M\right)^{1-\eta}}{1-\epsilon^{1-\eta}}\right]\frac{M^{1-\eta}}{a(T)\left(1-\eta\right)},
\end{equation}
given $a=B_0(T)/E_m$, the scaling exponent $\eta = 3/4$, and $m_0$ is the individual's mass at birth \cite{Yeakel2018}. 
Because the sharks that we are considering here are assumed to be ectotherms, we incorporate a temperature-dependence for metabolic parameters (Brown 2004), such that $B_0(T) = {\rm e}^C{\rm e}^{-E/kT}$, where the normalization constant $C=18.47$ for fish, the activation energy $E=0.63$ (eV), Boltzmann's constant $k=8.6173\times 10^{-5}$ (eV Kelvin ${}^{-1}$), and $T$ (Kelvin) is temperature \cite{Brown2004}.
Accordingly, shark individuals grow more quickly in warm environments, reaching the asymptotic mass $M$ at a younger age.

 
 
% For the time to reproduce, $t_{\lambda}=\tau\left(\epsilon_{\lambda}\right)$, where $\epsilon_{\lambda}$ is the fraction of the asymptotic mass where an organism is reproductively mature and should be close to one (typically $\epsilon_{\lambda}\approx0.95$; \cite{West:2001bv}). The growth rate is then given by $\lambda=\ln\left(\upsilon\right)/t_{\lambda}$ where $\upsilon$ is the number of offspring produced, and for any constant value of $\epsilon_{\lambda}$, this rate will scale as $\lambda\propto M^{\eta-1}$ for $M\gg m_{0}$ \cite{West:2001bv,moses2008rmo,gillooly2002esa,hou,Kempes:2012hy}.


% \noindent \textbf{Migration:}
In our model we consider two sites: a site where adult females give birth to juveniles (the nursery), and a site where individuals grow to their adult size (the adult site).
%Distance and velocity on migration rate d
The maximal migration rate is assumed to be a function of the distance between the nursery and the adult site, such that $d_{\rm max} = v/\delta$, where velocity $v=1~{\rm meters}\cdot {\rm sec}^{-1}$ and distance $\delta$ (meters) is varied. 
If the nursery is very close to the adult site, individuals will migrate between sites at a higher rate, such that both sites will be well mixed.
As the distance between sites increases, the extent to which the sites are mixed declines.
If the distance is large enough, individuals may grow in size as they travel between sites.

Migrations of adults to the nursery, and of both juveniles and adults from the nursery to the adult site, are considered separately because we assume these migrations operate according to two different drivers.
When newborns of size $m_0$ are born in the juvenile site, we assume they begin migrating to the adult site once they reach a size $m_j = (1/4)M$.
Assuming an asymptotic mass $M=110$ Kg, $m_0 = 8$ Kg, and $m_j = 27$ Kg, the time that it takes for this to occur is given by Eq. \ref{t1} and is $\tau = XX$.
Importantly, this migration is unlikely to be exact, and migration might occur at sizes smaller or larger than $m_j$.
To incorporate this additional flexibility into the juvenile migration, we assume that the migration rate $d$ is size-dependent, varying from $d=0$ below $m_0$ and increasingly sigmoidally to $d=d_j^{\rm max}$.
The \emph{mass-dependent} migration rate of individuals from the juvenile to adult site is thus described as
\begin{equation}
    d_j(m) = \frac{d_j^{\rm max}}{1 + {\rm exp}\{\frac{-\ell(m - m_j)}{M\xi_j}\}},
\end{equation}
where $\ell$ is the number of mass classes and $\xi_j$ describes the flexibility of the size-dependent migration: higher values of $\xi_j$ means that there is more variability in migrating juvenile size above and below $m_j$.
In words, as juveniles increase in size to $m_j$, their migration rate to the adult site increases sigmoidally to $d_{\rm max}$, whereas adults in the nursery by definition have already attained $d_{\rm max}$.
The flexibility of this migration increases with $\xi_j$, resulting in juveniles making their first migration to the adult site at both smaller and larger body sizes as $\xi_j$ increases.



%From adult site to nursery (a function of time)
We assume that females in the adult site migrate to the juvenile site seasonally.
Given a day of the year where the migration rate is maximized $d_a^{\rm max}$ on day $t_{\rm peak} = 180$ out of 365 days in a year, the migration rate decays before and after this peak in a Gaussian manner.
The \emph{time-dependent} migration rate of adults from the adult site to the juvenile site is thus described as
\begin{equation}
    d_a(t) = d_a^{\rm max}{\rm exp}\{\frac{-(t - t_{\rm peak})^2}{2\xi_a^2}\}.
\end{equation}
Here, $\xi_a$ describes the flexibility of the time-dependent migration: higher values of $\xi_a$ mean that there is more variability with regard to when adults migrate to the nursery. 
In words, adults migrate to the nursery during a specific time of year ($t_{\rm peak}$) and the flexibility of this migration increases with $\xi_a$, resulting in adults migrating at both earlier and later times as $\xi_a$ increases.


%Tooth drops
Because we aim to understand the shapes of tooth distributions from the perspective of shark population dynamics, we must simulate the loss of teeth over time in both nursery and adult sites.
We focus only on the loss of A1 and a1 teeth to reflect those used to build the empirical distributions.
To simulate accumulated tooth distributions, we assumed a tooth loss rate of one upper and lower tooth every 40 days, or $5.79\times 10^{-7}$ teeth/second.
Sharks of different body sizes drop teeth of different sizes, so shark life-history is directly captured by accumulated teeth of differing sizes.
The migrational dynamics serve to mix or segregate teeth dropped by reproducing adults and recently born offspring in the nursery site from the larger individuals in the adult site.


To compare simulated shark tooth distributions to those from different Eocene and contemporary sites, we first parameterized the model with known minimum winter and maximum summer mean ocean temperatures for nursery and adult sites, as well as the expected distance between sites.
In this sense, sites closer to the equator including Whiskey Bridge and Red Hot Truck Stop have more similar winter and summer temperature extremes, whereas sites closer to the poles, including Banks Island (NorthWest Territories, Canada) and Seymour Island (Antarctica), have larger differences in seasonal extremes.
Similarly, nursery and adult sites spanning a latitudinal gradient would be expected to have larger differences in seasonal temperature extremes, whereas those spanning longitudinal gradients would not.
While paleo-temperature extremes can be constrained based on independent climate indicators, the migrational windows $\xi_j$ and $\xi_a$ cannot.
As such, we simulated tooth distributions across a large range of $\xi_j$ and $\xi_a$ values, allowing us to assess which combination of values resulted in distributions sharing similarities to empirical distributions.
Because simulated distributions were both non-normal and multi-modal, to find parameterizations giving rise similar distributional shapes we numerically combined $K$ empirical and simulated distribution features for a given simulation -- including the means, standard-deviations, and both the presence/absence and values of one or multiple modes -- into a single error term $\epsilon(\xi_j,\xi_a) = \sum_{k=1}^K \left| w^{\rm sim}_k(\xi_j,\xi_a) - w^{\rm obs}_k \right|/w^{\rm obs}_k$ where $w^{\rm sim}_k(\xi_j,\xi_a)$ and $w^{\rm obs}_k$ are the measured values for the $k=1,...,K$ features described above for simulated and observed tooth distributions respectively, given the simulated migrational windows $(\xi_j,\xi_a)$.
Across a range of values for migrational windows, simulated distributions with a lower $\epsilon$ share greater similarities relative to the observed distribution, such that the $(\xi_j,\xi_a)$ resulting in ${\rm min}(\epsilon)$ was deemed the best match.
For each site we assumed no prior knowledge of whether it constituted a nursery or adult tooth distribution, and derived ${\rm min}(\epsilon)$ with respect to both simulated nursery and adult distributions, with the expectation that the correct site type would have the lower ${\rm min}(\epsilon)$.



%Parameterization

%Justification - SET UP
Examining equatorial (warm, low-seasonality) vs. non-equatorial (cold, high-seasonality) environments allows us to examine the effects of different temperatures regimes on accumulated tooth distributions, we also examined a system reflecting expected temperature regimes and seasonality of the Eocene Ocean.
We consider the dynamics of three different migrational systems: 
\emph{i}) a population with both adult and nursery sites far from the equator (cold temperatures, high seasonality),
\emph{ii}) a population with both adult and nursery sites near the equator (warm temperatures, low seasonality), and
\emph{iii}) a population with the adult site near the equator (warm, low seasonality), and the juvenile site far from the equator (cold, high seasonality).
[More about Eocene Ocean]
In addition we considered the effects of three different distances separating juvenile from adult sites, at 200, 400, and 650 Km.





\section{Results}

\subsection{Empirical tooth distributions}
A total of 1,053 anterior Sand Tiger teeth were measured across our four localities. 
The Banks Island collection consisted of 397 anterior teeth with a mean $\pm$ SD crown height of 13.70mm $\pm$ 3.41 (median = 14.10mm). 
The Seymour Island collections consisted of 126 anterior teeth with mean crown height of 17.13mm $\pm$ 6.57 (median = 17.13mm). 
The Red Hot Truck Stop collection included 372 anterior teeth with mean crown height of 11.60mm $\pm$ 3.43 (median = 11.78mm). 
The Whiskey Bridge collection included 158 anterior teeth with mean crown height of 22.51mm $\pm$ 4.59 (median = 22.55mm). 
A one-way ANOVA (df = 3, F = 283. 74, p<0.0001) resulted in a significant difference and a posthoc Tukey HSD analysis found significant differences between all pairwise comparisons of localities (p=0.001). 
We also tested the sample size sensitivity by bootstrap sampling each locality’s population for N > 3 over 1,000 iterations. 
We illustrate the bootstrap results, which depict the mean, minimum, and maximum for both the mean and standard error for the 1,000 iterations (Figs X and Y). 
In each locality, a sample size > 100 produced means and standard errors representative of the population. 

\subsection{Population model}
[General statement?]

%Mean
\textbf{Tooth distribution means} Tooth distribution means in both the nursery and adult site reflect the extent of mixing between nursery and adult sites.
When the migration of juveniles to the adult site is restricted (low $\xi_j$), tooth distribution means are far apart, with the nursery and adult site revealing lower and higher mean tooth sizes, respectively.
As the migration of juveniles increases (higher $\xi_j$), there is larger variation in the size classes entering the adult site for the first time such that the populations become increasingly mixed.
As a result, the mean of both nursery and adult site distributions begin to converge to an intermediate value.

The migration of adults to the nursery, which occurs during a time of year rather than with respect to a particular size class, impacts distribution means in a more complex way.
When adult migration is restrictive (low $\xi_a$) such that individuals migrate during a very short temporal window, fewer adults enter the nursery and comprise a smaller proportion of individuals, lowering the mean nursery tooth size.
When adult migration is very flexible (high $\xi_a$), migration occurs within a larger time span.
This means that the juveniles just entering the adult site are more likely to migrate back to the nursery before attaining larger body size, such that a significant number of adults migrating back the nursery site are smaller, inhibiting the increase in mean nursery tooth size, despite the influx of adults.

If the adult migration window is of intermediate magnitude, the mean nursery tooth size is maximized, of similar value to the adult site mean.
This increase in mean tooth size at the nursery is due to \emph{i}) the migration window being flexible enough to allow adults to accumulate at the nursery, but \emph{ii}) restrictive enough that juveniles arriving to the adult site have time to grow.
The cumulative effect is that the adults that do arrive at the nursery have attained greater size than when the adult migration window is large.

\textbf{Tooth distribution modality}
When the migration of juveniles to the adult site is flexible (high $\xi_j$), bimodality is observed at both sites if adult migration is restrictive, whereas unimodality is observed at both sites if adult migration is similarity flexible.
When both the migrational behavior of both juveniles and adults is flexible (Region XX), mixing between sites is maximized, and this serves to unify the tooth distributions of both sites under a single, unimodal, distribution.
As the adult migration window becomes more restrictive (lower $\xi_a$; Region XX), newly-arrived (smaller) adults remain at the adult site for a longer period of time, permitting growth and increasing the size differences between the nursery and adult site.
This creates a large enough difference in size distributions at the adult site, and -- as adult migrate back to the nursery -- the nursery site as well, introducing bimodality at both sites.

As the juvenile migration window becomes more restrictive, there are larger differences between the nursery and adult site modalities.
When both migrational behaviors are restrictive (Region XX), we observe that the nursery site in unimodal and the adult site is bimodal.
At the nursery site under these conditions, there are fewer adults and juveniles swamp the signal, resulting in unimodality.
At the adult site, less migration results in greater differentiation among larger adults and those newly-arrived, resulting in bimodality.

% If juvenile migration is restrictive but adult migration is flexible


% At both sites, when adult migration is restricted (low $\xi_a$), individuals remain in the adult site for a longer period of time, increasing the size differences between the nursery and adult site, and resulting in bimodality 

%1) Small adult mig. window: few adults at nursery, which becomes loaded with juvs
%2) Large adult mig. window: full adults and newly arrived juvs migrate into nursery
%3) Intermediate adult mig. window: restrictive enough that there juvs have time to get larger; flexible enough that enough adults can get to the nursery, and this promotes the mean value.

%Single vs. multiple modes

%Temperature impacts means of the distribution
%Migration windows impact the shape


\textbf{Reconstruction of paleo-ecologies}
Because our simulations of nursery/adult-site dynamics results in predictions of tooth distribution means, modalities (uni- or bimodal distributions), as well as the values of these modes, we can evaluate the most likely conditions giving rise to the empirical tooth distributions.


\section{Discussion}

%Distributions of shark teeth - useful from evo to eco
\textbf{Modern vs. Eocene Sand Tiger Sharks}
The modern analogue for the Eocene \emph{S. macrota} is the modern Sand Tiger \emph{C. taurus} based on the similarities in tooth shape throughout the entire dentition \cite{Cunningham2000}. In this study, we based body size distribution on anterior tooth crown height because the A1 and a1 are easy to distinguish and measure. In addition, limiting the positions measured from fossil teeth prevents potential for over representation of a single individual within the assemblage. We chose to represent shark size as anterior crown height to acknowledge that the regression for total length vs. tooth crown height established by Shimada (2002) may not be accurate for the Eocene \emph{S. macrota}. In fact, the maximum anterior tooth crown height we measured in the fossil localities exceeds the maximum length established for \emph{C. taurus}. 

The mean anterior tooth crown height for the modern Sand Tigers from Delaware Bay is 18.92 mm with a maximum of 26.11 mm, which corresponds with 213 cm and 295 cm, respectively. 
%Age and growth studies for modern sharks often estimate the asymptotic maximum length using the von Bertalanffy growth function, a generalized logistic model is L(a) = L_{\infty} (1 – e - k^{(a-a_{0} )} ) \end, where L is size, a is age, k is the growth coefficient, a_{0}\end is size at age zero. 
%Parameterization of the VBGF is by species and often there are differences between sexes and populations. 
%To discern accurate models, robust sample sizes are needed to correspond age and size (either mass or length).
A previous study for \emph{C. taurus} based on 96 individuals determined the asymptotic length for this species to be 296 cm\cite{Goldman2006}, which is similar to the maximum size at Delaware Bay.
In this study, the two marine fossil sites, La Meseta Fm. and Stone City Fm., include the largest anterior teeth measuring 41.00 mm and 32.57 mm, respectively. 
If the total length and anterior tooth crown heights are similar between extant and extinct Sand Tigers, then these anterior tooth crown heights correspond with total lengths of 486 cm and 381 cm at high and mid-latitude sites, respectively.
One explanation for larger Eocene Sand Tigers is that the regression to estimate total length from anterior tooth crown height for  \emph{C. taurus} is not applicable to the extinct \emph{S. macrota} due to differences in growth and maturity. 
Another possibility is that ancient Sand Tigers were larger than today's  \emph{C. taurus} due to warmer climate and/or a higher atmospheric carbon dioxide concentration \cite{Kim2020}.
%A previous study focusing on the La Meseta Fm. \emph{S. macrota} teeth found that their chemical composition supported climate simulations with carbon dioxide concentrations 3-6x pre-industrial levels \cite{Kim2020}. 

%Model Results
%The nuance of migration windows and their effects on distributions
\textbf{Factors Impacting Model Output}
A qualitative comparison of model results indicate that migration distance does not substantially affect body size distribution, but temperature and migration windows impact the median and modality of body size distribution. 
We expected the migration distance to cause greater differentiation in body size distributions between the nursery and adult sites; however, for the same temperature and migration regimes, the scaled density of tooth height distribution was similar regardless of 150, 400, or 1500 km between the nursery and adult sites. 
Therefore, we compared simulations from the 600 km migration distance to different temperature regimes matching the Eocene localities for which we had empirical data. 
We also noted that when migration windows became too large, the simulations were dominated by movement and computationally limited.

Sharks are ectotherms and therefore physiological features, such as growth rate and metabolic rate are temperature dependent (Eqn. 2.1). 
While temperature did affect growth rate and time to maturity, sharks are thought to have finite growth and an asymptotic length. 
Therefore, warmer temperatures result in faster growth and migration at a younger age since juvenile migration timing is mass-dependent in our dynamic model (Eqn. 2.2). 
Our initial model set the asymptotic size to that of extant \emph{C. taurus}; however, this resulted in tooth distributions substantially smaller than the empirical data collected from the four Eocene localities.
Therefore, in the Eocene simulations, we adjusted the asymptotic length to reflect the maximum anterior tooth crown height of 41mm, as represented in the La Meseta Fm. 

The parameters most influential in body size distribution were the flexibility in juvenile and adult migration windows. 
While the juvenile migration window is mass dependent, the adult migration window is based on time of year, which corresponds with the seasonal migration of many modern taxa.
Our simulations demonstrate that when juvenile migration is more strictly tied to a size threshold and the adult migration window limited, then the juvenile site will be dominated by smaller (i.e., younger) individuals because adults are restricted in their movement.
We expect habitats with strong seasonality in temperature, upwelling, or productivity to exhibit this scenario (i.e., high latitude regions).
In contrast, a stricter juvenile mass threshold for migration limits the incoming juveniles and therefore increases the mean body size at the adult site. 
We observed that increased movement from the nursery site (i.e., larger juvenile migration window) results in greater mixing and hence, a reduced mean body size at the adult site.

Nurseries are assumed for many species of extant sharks because they provide ample resources  and/or protection from predators. In most marine taxa, nurseries occur in estuaries and nearshore coastal regions \cite{beck2001identification}, but shark nurseries remain elusive to designate.

%Effects of temperature (high vs. low latitude)
%Make figure comparing within parameter change in modality/mean between different treatments

%VBGF paragraph to go some where
%The most frequently used approach to relate shark size and age is the von Bertalanffy growth function (VBGF). This generalized logistic model is L(a) = L (1 – e-k(a-a0)), where L is size, a is age, k is the growth coefficient, a0¬ is size at age zero. Parameterization of the VBGF is by species and often there are differences between sexes and populations. To discern accurate models, robust sample sizes are needed to correspond age and size (either mass or length). These data can be difficult to collect given that age determinations rely on counting vertebral rings, an intensive process that can be hampered because banding patterns are difficult to detect for small species or non-existent for species with less calcified vertebral structures. For larger species, accurate mass or length measurements can be a challenge to acquire and these data are important to predicting the asymptotic size, L. 

 \textbf{Model - Data comparisons of Body Size Distributions}
 
%What does the combination of data + model tell us about the Eocene?

[General statement about findings and what it means; MAB is bad at topic sentences]
\emph{Striatolamia macrota} show strikingly similar body size distributions to modern sand tiger sharks [I realize I may have the sp. mixed up], suggesting similar ecological constraints and behavior operated in the past as they do today.
It is expected that body size distributions within a shark species would show a normal distribution because sharks inhabit every ecological stage to adulthood (in contrast to mammals). 
Variations from this expectation indicate other processes.
However, body size distributions in extant sharks vary from normally distributed [e.g., Carcharodon charcarias, Hexanchus griseus, and Somniosus microcephalus (McClain et al. 2015)], to multi-modal [e.g., Cetorhinus maximus (McClain et al. 2015)], to right skewed [e.g., Megachasma pelagios (Watanabe and Papastamatiou 2019)], to left skewed [e.g., Otodus megalodon (Pimiento and Balk 2015)].
This model provides a testable hypotheses to the formation of multi-modal body size distributions, like the ones in this study and for Megachasma pelagios.

 Understanding biotic interactions is a challenge in the fossil record (but see Fraser et al. 2020; Sepkoski 2001; Govender 2021). 
 We compared simulation outputs to the empirical body size distributions from the modern Delaware Bay and Eocene deposits (Fig X).
 While we provide estimates of possible juvenile and adult migration windows to fit the empirical data, we think this approach is most powerful to discern qualitative trends and patterns rather than absolute interpretations.
For example, in the modern Delaware Bay site, there are parameter fits for both the juvenile and adult sites; however, the adult site distribution seems to be a better match.
Modern shark ecologists consider the summer nursery for juvenile \emph{C. taurus} to be the Plymouth, Kingston, Duxbury Bay off the Massachusetts coast \cite{kneebone2014movement}.
The mean total length for the Delaware Bay population sampled was 190 cm, which is the size at maturity for male \emph{C. taurus} \cite{gilmore1983reproduction}.
This correspondence in modern empirical body size distribution and what is known about this population with the dynamic population simulation is encouraging for interpreting broad scale patterns.

What can we discern about shark paleoecology by comparing empirical and simulated body size distributions?
In the mid latitude sites, the body size distributions of fossil Sand Tigers from Bashi and Tuscahoma Fms. suggest a nursury/juvenile site whereas the Stone City Fm. suggests an adult site (Fig X).
In the high latitude sites, there are plausible fits for both the juvenile and adults sites, but different trends for these localities.
The Banks Island sand tiger sharks indicate a relatively short adult migration window spanning 10-25, but the juvenile migration window possibilities span almost the entire range (Fig X).


\section{Conclusion}
The conclusion text goes here.

\vskip1pc

\ethics{Insert ethics text here.}

\dataccess{Insert data accessibility text here. '(If no information, then please include the text ``This article has no additional data'').}

\aucontribute{Insert author contributions text here (to be included if more than one author).}

\competing{Insert competing interests text here.}

\funding{Insert funding text here.}

\ack{Insert acknowledgment text here.}

\disclaimer{Insert disclaimer text here.}


\pagebreak

\section{Geologic Setting}

\subsection{Banks Island, NWT Canada}
At Banks Island, there is an abundance of coarsening-upward cycles within the Eureka SOund Formation that consist of shale, interbedded shale and silt, sand, then lignitic coal within the Cyclic Member \cite{Padilla2014}. 
Fossils are generally sparse through Banks Island, but an abundance of shark teeth, bivalves, and a trace fossil known as Ophiomorpha \cite{Miall1979, Padilla2014} have been recovered from the Cyclic Member. 
The unconsolidated sands that contained the shark teeth are dominated by fine to coarse-grained sand, but also include some pebble and conglomerate beds with clasts up to 12 cm \cite{Miall1979}). 
Miall (1979) concluded that the depositional environment was a proximal delta-front to delta-plain environment with various channels and coal swamps based on the lithology of the upward cycles of coal, shale, and sand in the Eureka Formation. 
The trace fossil known as Ophiomorpha found in the Cyclic Member \cite{Miall1979, Eberle2012} are inferred to be shrimp burrows suggesting a shallow-water, high-energy marine environment (Frey et al., 1978). 
The unconsolidated sand shark teeth were found in is interpreted as a channel or mouth bar deposit in the delta front (Padilla et al., 2014). 
A crocodyliform fossil found in the Cyclic Member on Banks Island (Eberle et al., 2014) suggests a mild temperature on Banks Island in the Eocene (Markwick, 1998). 
Oxygen isotope ratio analysis of biogenic phosphate from fossils on Ellesmere Island allowed Eberle et al. (2012) to estimate a mean annual temperature (MAT) of 8 ºC, and an annual range from 0-19 ºC. 
The paleo-precipitation has been estimated using isotopic analysis of fossil wood samples collected from the deltaic deposits in the Margaret Formation on Ellesmere Island and the Cyclic Member on northern Banks Island. High resolution $\delta^{13}{\rm C}$ values from tree ring samples were used to estimate annual precipitation and also indicated evergreen rather than deciduous trees in the Arctic (Schubert et al., 2012; Barbour et al., 2002). 
The results indicate a summer precipitation that was two to four times greater than that in winter – 1134mm in Summer compared to 366mm in Winter (see Schubert and Jahren., 2011; Equation 9). 
An ocean paleotemperature of 12-13 °C was estimated for the early-middle Eocene Arctic based on the TEX86 method (Sluijs et al., 2008). 
A riverine temperature on Ellesmere Island was estimated to be around 9 °C based on $\delta^{18}{\rm O}$ from terrestrial vertebrate bioapatite (Eberle et al., 2010). 
A mean paleosalinity of 12.7 PSU was estimated using a paleosalinity model modified by Kim et al. (2014); this is much lower than today’s Arctic surface waters, which have a salinity of 25-33 PSU and therefore implies a brackish water environment for the early Eocene Arctic Ocean (Kim et al., 2014).

\subsection{Seymour Island, Antarctica}
Seymour Island, Antartica
Seymour island is an andesitic, shallow succession of sandstone, siltstone, and shell marine beds and is stratified into 7 numbered units referred to as Tertiary Eocene La Meseta stratigraphic units (TELMs).
The La Meseta Fm. rests on top of the less-felsic upper Cretaceous- lower Paleocene Marambio Unit (Sadler, 1988; Marenssi et al., 1994; Ivany et al., 2008), and has undergone minimal burial and diagenetic alteration (Marenssi et al., 2002).  
TELMs are fault-bounded by an angular unconformity at the bottom of the formation, and biostratigraphically categorized (Sadler 1988; Long 1992; Reguero et al., 2012). 
La Meseta TELMs preserve fossil flora and fauna similar to temperate latitude species today and species living in temperate latitudes during the Eocene (Marenssi et al., 1994; Reguero et al., 2012). 
For example, sand tiger shark teeth (S. macrota) and sparnotheriodontid mammalian teeth (Victorlemoinea) are vertebrates also found in Brazil and Argentina (Marenssi, 1994).
Extant driftwood fossils suggest regular temperate rainfall in the region (Case, 1998; Ivany et al., 2008). 
Although the depositional setting could have potentially been influenced by freshwater influx, La Meseta faunal composition and geochemical analyses suggest normal marine conditions (Stilwell and Zinsmeister, 1992; Marenssi, 1998; Ivany et al., 2008).

Sand tiger shark teeth are limited to TELMs 2-5 (early Middle – late Middle Eocene), and absent from TELMs 6 and 7 (Long et al., 1992), which suggests a gradual cooling trend through the Late Eocene away from temperate conditions that would inhibit ability of sharks to survive at high latitudes (Kim et al., in review).
Oxygen isotope ratio analysis of biogenic carbonate from bivalve fossils in La Meseta Fm. corroborate this cooling trend over the course of the Middle Eocene, estimating a temperature change from ~15ºC (TELM 2) to ~10ºC (TELM 5; Dutton et al., 2002; Ivany et al., 2008). 
Oxygen isotope ratio analysis of biogenic phosphate from sand tiger shark fossils in La Meseta Fm. suggest temperatures ranging from ~12-16ºC during the same time range, but do not see a similarly conclusive cooling trend (Kim et al., in review).
Differences in temperature estimates may relate to biological differences between taxon (Kim et al., in review).

\subsection{Whiskey Bridge, Burleston County, TX USA}

The Whiskey Bridge locality lies within the late middle Eocene Crockett Formation, on top of the Sparta sand formation and is part of the late Middle Eocene Claiborne Group (Beard and Springer, 1999; Westgate, 2001; Harding et al., 2014; personal correspondence with C.J. Flis, Oct. 2016). 
It is often referred to as the “Main Glauconite Bed,” even though it is largely composed of fossiliferous, odinitic olive-green siliciclastic mudstone and sandstone and there is very little glauconite within the section (Breard and Springer, 1999; Westgate, 2001; Harding et al., 2014). 
Stone City Fm. has undergone minimal taphonomic alteration, and preserves one of the most diversified Middle Eocene vertebrate fauna within the Gulf Coastal Plain (Stanton and Nelson, 1980).
These diverse taxa include shallow neritic dwellers (i.e., gastropods, bivalves, ootolith-based taxa, rays, teleost fish, reptiles and sharks) and low to moderate diversity of foraminifera (Stanton and Nelson, 1980; Breard and Springer, 1999). 
The extant fauna is comparable to modern Gulf Coastal Plain fauna living in shallow inner shelf marine waters, and suggests that Stone City Fm. preserves a record of a tropical to sub-tropical climate with normal marine salinity (Breard and Springer, 1999; Harding et al., 2014; personal correspondence with C.J. Flis, Oct. 2016). 
Specifically, Stone City Fm. preserves three species of sand tiger sharks (Carcharias cuspidata, C. hopei, Striatolamia macrota; Breard and Springer, 1999). 
XRD and Mossbauer spectral analyses of Stone City Fm. clay pellets further support normal marine conditions and basic pH (7.5-8.5), based on the abundance of oodonite and paucity of glauconite (Harding et al., 2014), and suggest deposition in a shallower, tropical environment.

\subsection{Red Hot Truck Stop, Meridian, Mississippi}
The Tuscahoma Fm. includes about 110 meters of interbedded clay, silt, sand, and lignite, but only the upper ten feet is exposed at the locality (Mancini and Tew, 1995; Ingram, 1991). 
The sand and silt beds are laminar and cross-bedded, and range from 0.1 foot to 1.5 feet thick. At the base of the sand beds, fossiliferous channel lag deposits appear containing bioturbation, burrow casts, and concretions. 
Lignite is present throughout the Tuscahoma and overlying formations include several angiosperm pollen species, such as ferns and mosses that indicate a swamp and marsh environment (Mancini and Tew, 1995).
Palynofloras at the Red Hot Truck Stop locality contained 113 taxonomic groups that allowed an assessment of a paratropical vegetation habitat in the Gulf Coast (Harrington. 2003).
The early Eocene age is supported by mammalian fossil assemblage correlation (Beard and Dawson 2001) and pollen samples (Frederiksen 1998, Harrington 2003) and represents an early Eocene (early Wasatchian) age. 
The lithology of the Tuscahoma Formation and T4 Sand is consistent with that of a large-scale, fluvial-dominated deltaic system (Beard and Dawson, 2009). 
The large-scale cross bedding and cross-cutting represents the cut-and-fill depositional characteristics associated with estuarine channel facies (Ingram, 1991). 

Paleotemperature estimates indicate the early Eocene to have had the warmest climatic conditions in the Cenozoic Era (i.e., the last 66 million years; Keating-Bitonti et al., 2011). 
The shells of bivalve mollusks were analyzed for stable carbon and oxygen isotope ratios in the Bashi Formation on the Gulf Coast (ca. 54-52 Ma) at a paleolatitude of around 30°N (Keating-Bitonti et al., 2011). 
Ten shells were analyzed and resulted in a MAT (Mean Annual Temperature) of 〖26.5〗-+ 1.0 °C; 2-3 °C warmer than modern sea-surface MAT in the northern Gulf of Mexico (Keating-Bitonti et al., 2011; Levitus and Boyer, 1994). 
Analysis of mollusk shells from the Gulf Coast by Kobashi et al. (2001) found that the climate of the Mississippi Embayment (paleolatitude of 30°N) changed from a tropical environment of 26-27 °C in the Eocene, to paratropical, 22-23 °C in the Oligocene Epoch. 
Using modern regional salinity of 33 ppt, and the equation sought out by Grossman and Ku (1986), the estimated MAT of the Eocene Gulf Coast ocean water was approximately 23.3 °C, slightly cooler than the continental temperature (Kobashi et al., 2001). 

\pagebreak

%%%%%%%%%% Insert bibliography here %%%%%%%%%%%%%%
\bibliographystyle{RS}
\bibliography{aa_starving3}

\clearpage
\section{Supplementary Materials}




\end{document}
